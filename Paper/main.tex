% !TEX encoding = UTF-8
% !TEX program = pdflatex
% !TeX spellcheck = en_GB
% !BIB = biber


\documentclass[english]{article}
\usepackage{babel}
\usepackage[utf8]{inputenc}
\usepackage{graphicx}
\usepackage[obeyspaces]{url}
\graphicspath{{./images/}}
\usepackage{hyperref}
\hypersetup{
    colorlinks=true, 
	linkcolor=blue, 
	filecolor=blue, 
	citecolor = black,       
	urlcolor=blue, 
}
\usepackage{listings}
\lstset{
  xleftmargin=15pt,
  xrightmargin=0pt,
  framexleftmargin=0pt,
  framexrightmargin=0pt,
  basicstyle={\fontsize{9pt}{10pt}\ttfamily},
  columns=flexible,
  numbers=left,
  numbersep=10pt,
  numberstyle={\fontsize{9pt}{11pt}\selectfont\color[rgb]{0.4,0.4,0.1}},
  keepspaces=true,
  showstringspaces=false,
  identifierstyle=\color[rgb]{0.1,0.1,0.1},
  keywordstyle=\color{blue},
  commentstyle=\color[rgb]{0,0.3,0},
  morekeywords={rule, lemma},
  morekeywords=[2]{let, in},
  morekeywords=[3]{Fr, pk},
  morekeywords=[4]{In, Out},
  morekeywords=[5]{senc, aenc, h}
  morecomment=[s][keywordstyle3]{/*}{/},
  keywordstyle=\color[rgb]{0.44,0.57,0.65},
  stringstyle=\color{green},
  keywordstyle=[2]{\color[rgb]{0.86,0.57,0.18}},
  keywordstyle=[3]{\bfseries\color[rgb]{0,0.3,0.2}}
  }
\usepackage{biblatex}
\addbibresource{thud.bib}

\title{LTL+P Based Framework for Intrusion Detection}
\author{Zanolin Lorenzo}

\begin{document}

\maketitle

\begin{abstract}
The purpose of this paper is to introduce MONID, which is a framework created for system intrusion detection. This framework uses EAGLE, a rich and effectively monitorable logic, to express intrusion patterns using temporal logic formulas; EAGLE's ability to include data values and parameterized recursive equations makes it possible to represent security threats that include complex temporal event sequences and attacks with intrinsically statistical signatures succinctly. This tool can be used in offline and real-time scenarios. The implementation uses an algorithm for online monitoring that matches descriptions of the lack of an assault with indications of system execution; an alarm is set off whenever the standard is broken.
\end{abstract}

\tableofcontents
\newpage

\section{Introduction}
Even with all of the advances in computer security research, totally safe computer systems remain a long way off. Almost every large and complicated computer system nowadays contains vulnerabilities. Intrusion detection means maintaining constant surveillance on a system in order to detect any misuse of these weak areas as soon as feasible so that they can be repaired.

There are three Intrusion Detection System approaches, according to the literature: \textit{signature-based}\cite{gao2014cyber}, \textit{anomaly-based}\cite{jyothsna2011review} and \textit{hybrid}. The first approach aims to identify patterns and match them with known signs of intrusions relying on a database of previous intrusions. If activity within the network matches the “signature” of an attack or breach from the database, the detection system creates an alert. This approach has a low false-alarm rate, but it requires us to know the patterns of security attacks in advance and previously unknown attacks would go undetected. In contrast, \textit{anomaly-based} is capable to detect new attacks since an alarm is raised if an observed behavior deviates significantly from pre-learned normal behavior. Finally, a \textit{hybrid system} combines the best of both worlds by looking at patterns and one-off events, a Hybrid Intrusion Detection system can flag new and existing intrusion strategies.


\printbibliography

\end{document}