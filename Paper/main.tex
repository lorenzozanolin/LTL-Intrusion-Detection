% !TEX encoding = UTF-8
% !TEX program = pdflatex
% !TeX spellcheck = en_GB
% !BIB = biber

\documentclass[english]{article}
\usepackage{amsmath}
\usepackage{wasysym}
\usepackage{amssymb}
\usepackage{babel}
\usepackage[utf8]{inputenc}
\usepackage{graphicx}
\usepackage[obeyspaces]{url}
\graphicspath{{./images/}}
\usepackage{hyperref}
\hypersetup{
    colorlinks=true, 
	linkcolor=blue, 
	filecolor=blue, 
	citecolor = black,       
	urlcolor=blue, 
}
\usepackage{listings}
\lstset{
  xleftmargin=15pt,
  xrightmargin=0pt,
  framexleftmargin=0pt,
  framexrightmargin=0pt,
  basicstyle={\fontsize{9pt}{10pt}\ttfamily},
  columns=flexible,
  numbers=left,
  numbersep=10pt,
  numberstyle={\fontsize{9pt}{11pt}\selectfont\color[rgb]{0.4,0.4,0.1}},
  keepspaces=true,
  showstringspaces=false,
  identifierstyle=\color[rgb]{0.1,0.1,0.1},
  keywordstyle=\color{blue},
  commentstyle=\color[rgb]{0,0.3,0},
  morekeywords={rule, lemma},
  morekeywords=[2]{let, in},
  morekeywords=[3]{Fr, pk},
  morekeywords=[4]{In, Out},
  morekeywords=[5]{senc, aenc, h}
  morecomment=[s][keywordstyle3]{/*}{/},
  keywordstyle=\color[rgb]{0.44,0.57,0.65},
  stringstyle=\color{green},
  keywordstyle=[2]{\color[rgb]{0.86,0.57,0.18}},
  keywordstyle=[3]{\bfseries\color[rgb]{0,0.3,0.2}}
  }
\usepackage{biblatex}
\addbibresource{thud.bib}

\title{A Temporal Logic Based Framework for Intrusion Detection}
\author{Zanolin Lorenzo}

\begin{document}

\maketitle

\begin{abstract}
The purpose of this paper is to introduce MONID, which is a framework created for system intrusion detection. This framework uses EAGLE, a rich and effectively monitorable logic, to express intrusion patterns using temporal logic formulas; EAGLE's ability to include data values and parameterized recursive equations makes it possible to represent security threats that include complex temporal event sequences and attacks with intrinsically statistical signatures succinctly. This tool can be used in offline and real-time scenarios. The implementation uses an algorithm for online monitoring that matches descriptions of the lack of an assault with indications of system execution; an alarm is set off whenever the standard is broken.
\end{abstract}

\tableofcontents
\newpage

\section{Introduction}
Even with all of the advances in computer security research, totally safe computer systems remain a long way off. Almost every large and complicated computer system nowadays contains vulnerabilities. Intrusion detection means maintaining constant surveillance on a system in order to detect any misuse of these weak areas as soon as feasible so that they can be repaired.

There are three Intrusion Detection System approaches, according to the literature: \textit{signature-based}\cite{gao2014cyber}, \textit{anomaly-based}\cite{jyothsna2011review} and \textit{hybrid}\cite{garg2016hybrid}. The first approach aims to identify patterns and match them with known signs of intrusions relying on a database of previous intrusions. If activity within the network matches the “signature” of an attack or breach from the database, the detection system creates an alert. This approach has a low false-alarm rate, but it requires us to know the patterns of security attacks in advance and previously unknown attacks would go undetected. In contrast, \textit{anomaly-based} is capable to detect new attacks since an alarm is raised if an observed behavior deviates significantly from pre-learned normal behavior. Finally, a \textit{hybrid system} combines the best of both worlds by looking at patterns and one-off events, a Hybrid Intrusion Detection system can flag new and existing intrusion strategies. 

In this paper we will focus on \textit{signature-based} approach using temporal logic; we use EAGLE\cite{barringer2004rule,barringer2003eagle} to specify a system's attack-safe behavior. EAGLE allows recursively built temporal formulas, parameterizable by both logical formulas and data expressions, across three primitive modalities: "next", "previous", and "concatenation". The logic allows us to describe temporal patterns involving reasoning about data-values observed in individual events. Unlike LTL, EAGLE allowa us to design attacks with fundamentally statistical characteristics; password guessing attacks and ICMP-flood denial of service attacks are two examples. 
We'll see how the implementation, MONID, examines the event stream to see if the monitored formula (which is used to hold the pertinent summary of the system) is being broken.

%% INTRODURRE VELOCE ESEMPIO E STRUTTURA DEL PROGRAMMA

%%%%%   DA RISCRIVERE ALLA FINE
In Section 2 we will analyze the basics of EAGLE. In Section 3 we will analyze MONID framework and some evalutations on the performances. In Section 4 we will see some attacks specifications. In section 5 we will conclude.
%%%%5

\section{EAGLE, a Temporal Monitoring Logic}
According to~\cite{barringer2004program}, EAGLE offers a succinct but powerful set of primitives, essentially supporting recursive parameterized equations, with a minimal/maximal fix-point semantics together with three temporal operators: next-time ($\bigcirc$), previous-time ($\odot$), and concatenation ($\cdot$).
\subsection{Basics}
In EAGLE recursion definitions are supported; in the current framework we can build the following definitions:
\begin{align*}
& \underline{\text{min}}\ \text{Next}(\underline{\text{Form}}\ F) = \bigcirc F \\
& \underline{\text{max}}\ \text{Always}(\underline{\text{Form}}\ F) = F \land \bigcirc \text{Always}(F) \\
& \underline{\text{min}}\ \text{Eventually}(\underline{\text{Form}}\ F) = F \lor \bigcirc \text{Eventually}(F) \\
& \underline{\text{min}}\ \text{Until}(\underline{\text{Form}}\ F_1, \underline{\text{Form}}\ F_2) = F_2 \lor (F_1 \land \bigcirc \text{Until}(F_1,F_2))
\end{align*}

As we can see, rules are parameterized by an EAGLE formula (of type $\underline{\text{Form}}$), which means that we will be able to write EAGLE formulas such as $\text{Always}(\text{Eventually}(x>0))$. Also, the $\text{Always}$ operator is defined as maximal solution of the equation $X=F\land \bigcirc X$ and the $\text{Sometime}$ operator represents the minimal solution to the equation $X=F\lor \bigcirc X$.

%%  QUI MANCA LA PARTE DOVE SPIEGHI MAX, MIN, MON

Assume that we want to state the following property: "\textit{Whenever there is a login by any user x, then eventually the user x logs out}". In LTL we can write the following formula: 
\begin{align*}
    & \square{(\text{(action = login)}\rightarrow \underline{\text{let}}\ \text{k} =\ userId\ \underline{\text{in}}\ \lozenge(\text{action = logout}\land userId = k))}
\end{align*}
In this formula we use the operator \underline{let} \_ \underline{in} \_ to bind the value of $userId$ in the current event to the local variable $k$ whenever $\text{(action = login)}$ in the current event; we then impose the condition that the value of $userId$ in some event in future must be same as the user id bound to k and that the action of the event must be logout.
In EAGLE we can express the same property with the following rules:
\begin{align*}
    & \underline{\text{min}}\ \text{EvLogout}(\underline{\text{string}}\ k) = (\text{action = logout}\land userId = k) \lor \bigcirc \text{EvLogout}(k) \\
    & \underline{\text{mon}}\ M_2 = ((\text{action = login})\rightarrow \text{EvLogout}(userId)) 
\end{align*}
As a result, rules in EAGLE give us the power to create specific temporal operators as well as to bind and modify data. This property turns out to be crucial for succinctly expressing executions of attack-safe systems. To make the paper self-contained, we go back to the syntax and semantics of EAGLE.

Lastly, two assumptions must be made:\begin{enumerate}
    \item There is a finite sequence of occurrences called $\sigma=\alpha_{1},\ldots ,\alpha_{n}$ that is a merge of the system registered logs organized by ascending time. 
    The structure of an event record $\alpha_i$ is the following:
    \begin{align*}
        & \text{LoginLogoutEvent}\{userId:\underline{\text{string}},action: \underline{\text{int}}, time: \underline{\text{double}}\} 
    \end{align*}
    An example of event could be: $\{userId:\text{"Lori"},action:\text{login},time:\text{20}\}$
    \item For each attack, there is a formula $F$ which specifies the absence of it.
\end{enumerate}

\subsection{Syntax and Semantics}
\textbf{Syntax.}
A specification $S$ is made up of an observer part $O$ and a declaration part $D$. $O$ is made up of zero or more monitor definitions $M$, which define what will be watched, while $D$ is made up of zero or more rule definitions $R$. The names of rules and monitors are $(N)$.
\begin{align*}
    S &::= D \, O \\
    D &::= R^* \\
    O &::= M^* \\
    R &::= \{\underline{\text{max}}\ |\ \underline{\text{min}} \}\ N(T_1\ x_1, \ldots, T_n\ x_n) = F \\
    M &::= \underline{\text{mon}}\ N = F \\
    T &::= \underline{\text{Form}}\ |\ \text{primitive type} \\
    F &::= \text{expression}\ |\ \underline{\text{true}}\ |\ \underline{\text{false}}\ |\ \neg F\ |\ F_1 \land F_2\ |\ F_1 \lor F_2\ |\\
            &F_1 \rightarrow F_2\ |\ \bigcirc F\ |\ \odot F\ |\ F_1 \cdot F_2\ |\ N(F_1, \ldots, F_n)\ |\ x_i
\end{align*}
The types of parameters include primitive types like \underline{int}, \underline{long}, \underline{float}, etc. or formulas of type \underline{Form}.
A term indicating whether the interpretation is maximal ($\underline{\text{max}}$) or minimal ($\underline{\text{min}}$) comes before a rule specification $R$. While minimal rules establish \textit{liveness properties} (something good eventually happens), maximal rules define \textit{safety properties} (nothing bad ever happens); as we will see, the difference becomes important only when we are evaluating the formula in specific instants. Recursive definitions of rules are permitted as long as they are tightly guarded by a temporal operator. Finally,the definitions starting with keyword $\underline{\text{mon}}$ specifies the EAGLE formulas to be monitored and cannot have a recursive definition. We shall see that monitored rules will evolve as new events appear.

\textbf{Semantics.} The semantics of the logic is defined from the \textit{satisfaction} relation $\vDash$ which defines whether an execution trace $\sigma$ satisfy the specification $\varphi$. The $i$'th state $s_i$ of a trace $\sigma$ is denoted by $\sigma(i)$. The term $\sigma[i, j]$ denotes the sub-trace of $\sigma$ from position $i$ to position $j$, both positions included; if $i \geq j$ then $\sigma[i, j]$ denotes the empty trace.

Given a trace $\sigma$ and a specification $D \, O$, satisfaction is defined as follows:
\[
\sigma \models D \, O \text{ iff } \forall (\underline{\text{mon}}\ N = F) \in O.\ \sigma ,1 \vDash_D \, F
\]
In other words, a trace satisfies a specification if it fulfills each monitored formula when monitoring from position 1, which is the initial state.



\subsection{Relationship to Other Logics}
%% RAPPORTI CON LE ALTRE LOGICHE?????
According to~\cite{barringer2004rule},

We will use it to create formulas that describe the absence of some attacks and given a trace of events $\sigma$ generated from the logs we will check whether 

\section{MONID: Structure analysis}

\section{Example of attacks}

\section{Conclusions}

\printbibliography

\end{document}